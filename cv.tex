\documentclass[12pt, a4paper]{article}

% Use sans-serif font 
\renewcommand{\familydefault}{\sfdefault}

% Change geometry
\usepackage{geometry}

% Clickety-click things
\usepackage{hyperref}

% Internet symbols of fixed width
\usepackage[fixed]{fontawesome5}

% Colours
\usepackage{xcolor}

% CMSS (computer modern sans serif) does not define bullet 
\usepackage{textcomp}

% Customise itemize, enumerate, list...
\usepackage{enumitem}

% Make margins wider
\geometry{left=15mm, right=15mm, top=15mm, bottom=15mm}

% Remove page numbers
\pagenumbering{gobble}

\hypersetup{
	% Remove box from links
	colorlinks=true,
	% Colour of url links
	urlcolor=textcolour
}

% Set up colours to use
\definecolor{themecolour}{rgb}{0.858, 0.188, 0.478}
\definecolor{textcolour}{HTML}{373D3F}
\color{textcolour}

% No indent on itemize
\setlist[itemize]{leftmargin=*,
		  topsep=1.5mm,
		  parsep=1.5mm}

%------------------------------------------------------------------------------
% Formatting of name
\newcommand{\name}[1]{%
	\begin{center}
        \huge{\textbf{\uppercase{#1}}}
	\end{center}\vspace{-1mm}\par
}
%------------------------------------------------------------------------------

%------------------------------------------------------------------------------
% Create commands to capture CV parameters
\newcommand*{\email}[1]{\def\email{#1}}
\newcommand*{\github}[1]{\def\github{#1}}
\newcommand*{\linkedin}[1]{\def\linkedin{#1}}
\newcommand*{\location}[1]{\def\location{#1}}
\newcommand*{\phone}[1]{\def\phone{#1}}
\newcommand*{\stackoverflow}[1]{\def\stackoverflow{#1}}
\newcommand*{\website}[1]{\def\website{#1}}
%------------------------------------------------------------------------------

%------------------------------------------------------------------------------
% Card entries have a font awesome icon (#1) and associated text (#2)
\newcommand*{\cardentry}[2]{\faThemeIcon{#1}\hspace{\icontxtspace}#2}

% Colour font-awesome icons with CV theme colour
\newcommand*{\faThemeIcon}[1]{\textcolor{themecolour}{\faIcon{#1}}}

% Define length to separate font awesome icon from associated text
\newlength{\icontxtspace}
\setlength{\icontxtspace}{.25em}

% Make entry for email address
\newcommand*{\emailentry}{%
    \cardentry{envelope}%
              {\href{mailto:\email}%
                    {\email}}
}

% Make entry for website
\newcommand*{\siteentry}{%
    \cardentry{globe-europe}%
              {\href{https://\website}%
                    {\website}}
}

% Make entry for location
\newcommand*{\locationentry}{%
    \cardentry{map-marker-alt}%
              {\location}
}

% Make entry for linkedin
\newcommand*{\linkedinentry}{%
    \cardentry{linkedin}%
              {\href{https://www.linkedin.com/in/\linkedin/}%
                    {linkedin.com/in/\linkedin}}
}

% Make entry for stackoverflow
\newcommand*{\stackoverflowentry}{%
	\cardentry{stack-overflow}%
	          {\href{https://stackoverflow.com/users/\stackoverflow?tab=profile}%
                    {user:\stackoverflow}}
}

% Make entry for phone number
\newcommand*{\phoneentry}{%
    \cardentry{phone}%
              {\phone}
}

% Make entry for GitHub account
\newcommand*{\githubentry}{%
    \cardentry{github}%
              {\href{https://github.com/\github}%
                    {\github}}
}
% Command to create informational header with contact details and social info
\newcommand*{\makecard}{%
	\hspace*{\fill}
	\siteentry \hfill \stackoverflowentry \hfill \githubentry
	\hspace*{\fill} \vspace{3pt}\\
	\hspace*{\fill}
	\emailentry \hfill \phoneentry \hfill \locationentry
	\hspace*{\fill} \par
}
%------------------------------------------------------------------------------

%------------------------------------------------------------------------------
% Formatting of statement
\newcommand{\statement}[1]{%
	\parindent0pt
	\begin{center}
		``\textsl{#1}''
	\end{center}
    \vspace{-0.4cm}
}
%------------------------------------------------------------------------------

%------------------------------------------------------------------------------
% Define command for CV sections
\newcommand{\cvsection}[1]{%
	\parindent0pt
	\bigskip\color{themecolour}{\huge{#1}}
	\vspace{-7pt}\\
	\themerule\par
}

% Define horizontal rule to use after sections
\newcommand*{\themerule}{%
	\color{themecolour}\rule{\textwidth}{1.75pt}\color{textcolour}
}
%------------------------------------------------------------------------------

%------------------------------------------------------------------------------
% Higher level macros for formatting CV entries

% Use as \eduitem{Institution}{Qualification}{Date}
\newcommand{\eduitem}[3]{\datedfaitem{#1}{\faIcon{graduation-cap}}{#2}{#3}}

% Use as \awarditem{Event}{Award}{Date}
\newcommand{\awarditem}[3]{\datedfaitem{#1}{\faIcon{trophy}}{#2}{#3}}

% Use as \jobitem{Role}{Company}{Date}
\newcommand{\jobitem}[3]{\datedfaitem{#1}{\faIcon{briefcase}}{#2}{#3}}

% Use as \confitem{Location}{Title}{Date}
\newcommand{\confitem}[3]{\datedfaitem{#1}{}{#2}{#3}}

% Use as \talkitem{Event}{Title}{Date}
\newcommand{\talkitem}[3]{\datedfaitem{#1}{\faIcon{microphone}}{#2}{#3}}

% Use as \teachitem{Content}
\newcommand{\teachitem}[1]{\faitem{\faIcon{chalkboard-teacher}}{#1}}

% Use as \quantitem{Content}
\newcommand{\quantitem}[1]{\faitem{\faIcon{chart-line}}{#1}}
%------------------------------------------------------------------------------

%------------------------------------------------------------------------------
% Lower level macros for formatting CV entries

% Use as \datedfaitem{title}{faicon}{details}{date}
\newcommand{\datedfaitem}[4]{%
    \textbf{#1}\par
    #2 \textsl{#3} \hfill \faIcon[regular]{calendar-alt} \textsl{#4}\vspace{2mm}\par
}

% Use as \faitem{faIcon}{details}
\newcommand{\faitem}[2]{%
	\begin{itemize}[
		align=left,
		leftmargin=2em,
  		itemindent=0pt,
  		labelsep=0pt,
 		labelwidth=2em]
	\item[{#1}] #2
	\end{itemize}
}
%------------------------------------------------------------------------------


\phone{+44 (0)7834 xxx xxx}
\locationurl{https://goo.gl/maps/bEk6nHTSkGxsFm4y6}
\location{Newcastle upon Tyne, UK}
\email{jwalton3141@gmail.com}
\linkedin{jwalton93}
\website{jwalton.info}
\stackoverflow{11021886}
\github{jwalton3141}

\begin{document}



\name{Dr. Jack Walton}

\makecard

\statement{PhD educated and analytical problem solver seeks challenging and
           intellectually stimulating work}



\cvsection{Education}

\eduitem{University of Newcastle upon Tyne, UK}%
        {PhD, Applied Mathematics\\
         \href{https://jwalton.info/assets/thesis.pdf}%
              {{\footnotesize\faIcon{paperclip}}``Bayesian Inference for Models of Collective Behaviour''}}%
        {2016--2020}

\eduitem{University of Newcastle upon Tyne, UK}%
        {Master of Mathematics, 1st Class}%
        {2012--2016}



\cvsection{Experience}

\jobitem{Data Engineer \& Trainer}%
        {\href{https://www.jumpingrivers.com/}{Jumping Rivers Ltd.}}%
        {2020--Present}

\vspace{-1em}

\cvsubsection{Notable Client Projects}

\faitem{\faIcon{aws}}{3.25em}%
       {\emph{Serverless \& massively scalable web app with AWS\hfill{}Lead Developer}

       \begin{adjustwidth}{0em}{3em}

       \sloppy{}Used Docker to productionise a Bayesian network model
       (implemented in R), parameterised PDF report and data visualisation generation,
       via AWS Lambda and API Gateway
       (\href{https://github.com/nationalarchives/diagram/tree/live/api}%
             {\texttt{github.com/nationalarchives/diagram/tree/live/api}}).

       Developed Terraform for IaC, provisioning development, staging and production environments
       (\href{https://github.com/nationalarchives/DiAGRAM-terraform/}%
             {\texttt{github.com/nationalarchives/DiAGRAM-terraform}}).

       Automated front and backend deployment, controlling application life
       cycle with GitHub and GitHub Actions.

       \end{adjustwidth}}%

\faitem{\faIcon{microsoft}}{3.25em}%
       {\emph{Azure Databricks \& App Service for ETL\hfill{}Lead Developer}

       \begin{adjustwidth}{0em}{3em}

       Administered Databricks and prepared init scripts for R workloads.

       Adapted existing ETL R scripts to write results to blob storage via
       \href{https://cran.r-project.org/web/packages/AzureStor/vignettes/intro.html}%
            {\{AzureStor\}}.

       Improved typical application deploy time from $\sim$20mins to $\sim$2--3mins.

       \end{adjustwidth}}%

\vspace{-.4em}

\cvsubsection{Responsibilities \& Achievements}

\begin{itemize}[
    itemsep=.2em,
    labelwidth=2em,
    leftmargin=3.25em,
    topsep=0em]

     \item[\faIcon{layer-group}]%
       {Training tech stack product owner; prioritised, performed and delegated development and
       maintenance work to support delivery of JR training courses and acssociated training
       materials.}

     \item[\faIcon{python}]%
       {Taught over 250 hours of Python programming courses, with course difficulty
        ranging from ``what is a variable?'' to machine learning with \texttt{sklearn}.}%

     \item[\faIcon{chalkboard-teacher}]%
       {Developed and delivered courses on the probabilistic programming
        language Stan, data visualisation in Python, best-coding practices and how to
        implement them, and getting the most out of Posit (RStudio) Connect as a data
        scientist.}%

    \item[\faIcon{cloud}]%
       {Employed Ansible and Terraform to manage, provision and configure client and
        internal infrastructure, working with a number of different cloud providers.}%

    \item[\faIcon{r-project}]%
       {Collaborated on the development of a suite of R packages to standardise and
        simplify internal project management and automate recurring tasks.}

\end{itemize}

\clearpage




\cvsection{Skills}

\vspace{-.5em}
\begin{center}
    \begin{tabular}{@{}*{10}{c?}@{}}
        & Python & R &  Stan & bash & git & Ansible & Terraform & Docker & Packer
    \end{tabular}
\end{center}

\begin{itemize}[
    itemsep=.2em,
    labelwidth=2em,
    leftmargin=2em,
    topsep=0em]

\item[\faIcon{bullhorn}]%
     {Confident and competent speaker; adept at communicating technical concepts to
      non-technical audiences.}%

\item[\faIcon{linux}]%
     {Capable of performing all manner of duties from a CLI, scripting in bash, and
      using standard command-line tools such as \texttt{ssh}, \texttt{grep},
      \texttt{sed}, \texttt{vi}, etc.}%

\item[\faIcon{git}]%
     {``Intermediate'' git user, harnessing version control on all code-based
      projects, and using CI/CD tools to aid code stability, dictate release cycles,
      and enforce good-coding practices.}%

\end{itemize}



\cvsection{Community}

\begin{itemize}[
    itemsep=.2em,
    labelwidth=2em,
    leftmargin=2em,
    topsep=0em]

\item[\faIcon{users}]%
     {Chief-organiser and chair of
      \href{https://www.meetup.com/newcastle-upon-tyne-data-science-meetup/}%
           {North East Data Science meetup series},
      overseeing $\sim$50\% growth in group membership. Co-organiser of
      \href{https://www.meetup.com/north-east-databricks-meetup/}%
           {North East Databricks meetup series.}}

\item[\faIcon{stack-overflow}]%
     {Regular contributor on Stack Overflow---answering over 130 questions,
      and reaching $>$275k users.}%

\item[\faIcon{mouse-pointer}]%
     {Maintainer of \href{https://\website}{\website}, a programming blog and
      personal website which attracts 800--1000 unique monthly users, with
      $\sim$85\% of traffic acquired through organic search.}

\end{itemize}

\cvsection{Awards}

\awarditem{Dynamites22}%
          {\href{https://www.dynamonortheast.co.uk/dynamites22-shortlist-unveiled/}%{
                {Rising Star (shortlist)}}%
          {November 2022}

\awarditem{Northumbrian Water Hackathon}%
          {Best team solution}%
          {July 2019}

\awarditem{UK Conference on Multiscale Biology}%
          {Best poster award}%
          {April 2018}

\awarditem{EPSRC}%
          {PhD project funding (3.5 years)}%
          {2016--2020}

\end{document}
