%------------------------------------------------------------------------------
% Formatting of name
\newcommand{\name}[1]{%
	\begin{center}
        \huge{\textbf{\uppercase{#1}}}
	\end{center}\vspace{-1mm}\par
}
%------------------------------------------------------------------------------

%------------------------------------------------------------------------------
% Create commands to capture CV parameters
\newcommand*{\email}[1]{\def\email{#1}}
\newcommand*{\github}[1]{\def\github{#1}}
\newcommand*{\linkedin}[1]{\def\linkedin{#1}}
\newcommand*{\location}[1]{\def\location{#1}}
\newcommand*{\phone}[1]{\def\phone{#1}}
\newcommand*{\stackoverflow}[1]{\def\stackoverflow{#1}}
\newcommand*{\website}[1]{\def\website{#1}}
%------------------------------------------------------------------------------

%------------------------------------------------------------------------------
% Card entries have a font awesome icon (#1) and associated text (#2)
\newcommand*{\cardentry}[2]{\faThemeIcon{#1}\hspace{\icontxtspace}#2}

% Colour font-awesome icons with CV theme colour
\newcommand*{\faThemeIcon}[1]{\textcolor{themecolour}{\faIcon{#1}}}

% Define length to separate font awesome icon from associated text
\newlength{\icontxtspace}
\setlength{\icontxtspace}{.25em}

% Make entry for email address
\newcommand*{\emailentry}{%
    \cardentry{envelope}%
              {\href{mailto:\email}%
                    {\email}}
}

% Make entry for website
\newcommand*{\siteentry}{%
    \cardentry{globe-europe}%
              {\href{https://\website}%
                    {\website}}
}

% Make entry for location
\newcommand*{\locationentry}{%
    \cardentry{map-marker-alt}%
              {\location}
}

% Make entry for linkedin
\newcommand*{\linkedinentry}{%
    \cardentry{linkedin}%
              {\href{https://www.linkedin.com/in/\linkedin/}%
                    {linkedin.com/in/\linkedin}}
}

% Make entry for stackoverflow
\newcommand*{\stackoverflowentry}{%
	\cardentry{stack-overflow}%
	          {\href{https://stackoverflow.com/users/\stackoverflow?tab=profile}%
                    {user:\stackoverflow}}
}

% Make entry for phone number
\newcommand*{\phoneentry}{%
    \cardentry{phone}%
              {\phone}
}

% Make entry for GitHub account
\newcommand*{\githubentry}{%
    \cardentry{github}%
              {\href{https://github.com/\github}%
                    {\github}}
}
% Command to create informational header with contact details and social info
\newcommand*{\makecard}{%
	\hspace*{\fill}
	\siteentry \hfill \stackoverflowentry \hfill \githubentry
	\hspace*{\fill} \vspace{3pt}\\
	\hspace*{\fill}
	\emailentry \hfill \phoneentry \hfill \locationentry
	\hspace*{\fill} \par
}
%------------------------------------------------------------------------------

%------------------------------------------------------------------------------
% Formatting of statement
\newcommand{\statement}[1]{%
	\parindent0pt
	\begin{center}
		``\textsl{#1}''
	\end{center}
    \vspace{-0.4cm}
}
%------------------------------------------------------------------------------

%------------------------------------------------------------------------------
% Define command for CV sections
\newcommand{\cvsection}[1]{%
	\parindent0pt
	\bigskip\color{themecolour}{\huge{#1}}
	\vspace{-7pt}\\
	\themerule\par
}

% Define horizontal rule to use after sections
\newcommand*{\themerule}{%
	\color{themecolour}\rule{\textwidth}{1.75pt}\color{textcolour}
}
%------------------------------------------------------------------------------

%------------------------------------------------------------------------------
% Higher level macros for formatting CV entries

% Use as \eduitem{Institution}{Qualification}{Date}
\newcommand{\eduitem}[3]{\datedfaitem{#1}{\faIcon{graduation-cap}}{#2}{#3}}

% Use as \awarditem{Event}{Award}{Date}
\newcommand{\awarditem}[3]{\datedfaitem{#1}{\faIcon{trophy}}{#2}{#3}}

% Use as \jobitem{Role}{Company}{Date}
\newcommand{\jobitem}[3]{\datedfaitem{#1}{\faIcon{briefcase}}{#2}{#3}}

% Use as \confitem{Location}{Title}{Date}
\newcommand{\confitem}[3]{\datedfaitem{#1}{}{#2}{#3}}

% Use as \talkitem{Event}{Title}{Date}
\newcommand{\talkitem}[3]{\datedfaitem{#1}{\faIcon{microphone}}{#2}{#3}}

% Use as \teachitem{Content}
\newcommand{\teachitem}[1]{\faitem{\faIcon{chalkboard-teacher}}{#1}}

% Use as \quantitem{Content}
\newcommand{\quantitem}[1]{\faitem{\faIcon{chart-line}}{#1}}
%------------------------------------------------------------------------------

%------------------------------------------------------------------------------
% Lower level macros for formatting CV entries

% Use as \datedfaitem{title}{faicon}{details}{date}
\newcommand{\datedfaitem}[4]{%
    \textbf{#1}\par
    #2 \textsl{#3} \hfill \faIcon[regular]{calendar-alt} \textsl{#4}\vspace{2mm}\par
}

% Use as \faitem{faIcon}{details}
\newcommand{\faitem}[2]{%
	\begin{itemize}[
		align=left,
		leftmargin=2em,
  		itemindent=0pt,
  		labelsep=0pt,
 		labelwidth=2em]
	\item[{#1}] #2
	\end{itemize}
}
%------------------------------------------------------------------------------
